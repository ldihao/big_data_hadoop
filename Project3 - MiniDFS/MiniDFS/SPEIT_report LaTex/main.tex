\documentclass[10pt]{article}
\usepackage[english]{babel}
\usepackage{amsmath}
\usepackage{graphicx}
\usepackage[colorinlistoftodos]{todonotes}
\pagestyle{headings}
\usepackage{indentfirst}
\usepackage[utf8]{inputenc}
\usepackage{xeCJK}
\usepackage{float}

%% Code blocl setting
\usepackage{listings}
\renewcommand{\lstlistingname}{Code}% Listing -> Algorithm
\usepackage{color}

\definecolor{dkgreen}{rgb}{0,0.6,0}
\definecolor{gray}{rgb}{0.5,0.5,0.5}
\definecolor{mauve}{rgb}{0.58,0,0.82}

\lstset{frame=tb,
  language=Python,
  aboveskip=3mm,
  belowskip=3mm,
  showstringspaces=false,
  columns=flexible,
  basicstyle={\small\ttfamily},
  numbers=none,
  numberstyle=\tiny\color{gray},
  keywordstyle=\color{blue},
  commentstyle=\color{dkgreen},
  stringstyle=\color{mauve},
  breaklines=true,
  breakatwhitespace=true,
  tabsize=3
}

\begin{document}

\begin{titlepage}

\newcommand{\HRule}{\rule{\linewidth}{0.5mm}} % Defines a new command for the horizontal lines, change thickness here

\center % Center everything on the page

%----------------------------------------------------------------------------------------
%	HEADING SECTIONS
%----------------------------------------------------------------------------------------

\textsc{\LARGE Shanghai Jiaotong University}\\[1.5cm] % Name of your university/college
\textsc{\Large Big Data Processing Technology}\\[0.5cm] % Major heading such as course name
%\textsc{\large Minor Heading}\\[0.5cm] % Minor heading such as course title

%----------------------------------------------------------------------------------------
%	TITLE SECTION
%----------------------------------------------------------------------------------------

\HRule \\[0.4cm]
{ \huge \bfseries Project 3: Mini DFS}\\[0.4cm] % Title of your document
\HRule \\[1.5cm]

%----------------------------------------------------------------------------------------
%	AUTHOR SECTION
%----------------------------------------------------------------------------------------

\begin{minipage}{0.4\textwidth}
\begin{flushleft} \large
\emph{Author:}\\
% Your name
LUO Dihao \\118260910039\\ % Your name
YAN Shuhan \\118260910050\\ % Your name
SHEN Shengyang \\118260910042\\
\end{flushleft}
\end{minipage}
~
\begin{minipage}{0.4\textwidth}
\begin{flushright} \large
\emph{Supervisor:} \\
Patrick  \textsc{BERBON} % Supervisor's Name
\end{flushright}
\end{minipage}\\[2cm]

% If you don't want a supervisor, uncomment the two lines below and remove the section above
%\Large \emph{Author:}\\
%John \textsc{Smith}\\[3cm] % Your name

%----------------------------------------------------------------------------------------
%	DATE SECTION
%----------------------------------------------------------------------------------------

{\large \today}\\[2cm] % Date, change the \today to a set date if you want to be precise

%----------------------------------------------------------------------------------------
%	LOGO SECTION
%----------------------------------------------------------------------------------------

\includegraphics[width=0.5\textwidth]{logo_SPEIT.jpg}\\[1cm] % Include a department/university logo - this will require the graphicx package

%----------------------------------------------------------------------------------------

\vfill % Fill the rest of the page with whitespace

\end{titlepage}
\indent

\section{Introduction}

Our mini-DFS (Distributed File System) is based on Java and is composed by following files:

\begin{itemize}
  \item Main
  \begin{itemize}
    \item Main.java : the client interface
    \item Manager.java : we use this class to store and share basic information among nodes
  \end{itemize}
  \item Node
  \begin{itemize}
    \item DataNode.java : datanode class to perform save/read/recover functions
    \item NameNode.java : namenode class to perform load/ls/split/read/fetch functions
  \end{itemize}
  \item Tools
  \begin{itemize}
    \item Block.java : class to store file block's information
    \item FileHelper.java : class to read/write/recover file blocks
    \item FileMap.java : class to store file mapping information
    \item MyFile.java : class to store file information (containing several blocks)
    \item Operation.java : class storing different operations including ls,put,fetch,read,quit,recover
  \end{itemize}
\end{itemize}

\subsection{Usage}

Users can access Mini-DFS by directly running Main.java.

\begin{itemize}
  \item ls : show list of files
  \item put source\_file\_path : upload local file to mini-DFS
  \item read file\_ID : read the first block of required file
  \item fetch file\_id save\_path : download file from mini-DFS to local file system
  \item recover file\_id : try to recover file block if some datanode lost file blocks
  \item quit : exit Mini-DFS
\end{itemize}


\section{Architecture}

\begin{figure}[H]
\centerline{\includegraphics[width = 1\textwidth]{screenshot//process.png}}
\caption{Architecture}
\label{fig_process}
\end{figure}

As we can see from the image, when a client type some command to the interface \textit{Main}, \textit{Main} share this command with \textit{NameNode} and \textit{DataNode} through \textit{Manager}. At the same time, we use \textit{java.util.concurrent.CyclicBarrier} \textit{name\_event}, \textit{main\_event}, \textit{data\_event} to send signal from \textit{Main} to \textit{NameNode} and \textit{DataNode}. So, then \textit{Main} uses name\_event.await() to notify \textit{NameNode} to do command, \textit{NameNode} uses data\_event.await() to notify \textit{DataNode} to do command, finally \textit{DataNode} uses main\_event.await() to notify \textit{Main} that all process has been done and it can receive next command from client.

\section{Example}

First, we upload a file to Mini-DFS using \textit{put}.

\begin{figure}[H]
\centerline{\includegraphics[width = 1\textwidth]{screenshot//put_01.png}}
\caption{Usage: put}
% \label{fig_process}
\end{figure}

In directory \textit{dfs}, we can see it creates seven blocks distributed uniformly among four datanodes.

\begin{figure}[H]
\centerline{\includegraphics[width = 1\textwidth]{screenshot//put_02.png}}
\caption{Usage: put}
% \label{fig_process}
\end{figure}

Then, we read the first block of this file. Since it's of format pdf, it contains mostly unreadable strings.

\begin{figure}[H]
\centerline{\includegraphics[width = 1\textwidth]{screenshot//read_01.png}}
\caption{Usage: read}
% \label{fig_process}
\end{figure}

Next, we download this file to local file system: directory \textit{./dfs}

\begin{figure}[H]
\centerline{\includegraphics[width = 1\textwidth]{screenshot//fetch_01.png}}
\caption{Usage: fetch}
% \label{fig_process}
\end{figure}

It appears in the file system.

\begin{figure}[H]
\centerline{\includegraphics[width = 1\textwidth]{screenshot//fetch_02.png}}
\caption{Usage: fetch}
% \label{fig_process}
\end{figure}

Then, we delete \textit{datanode0} and try to recover it.

\begin{figure}[H]
\centerline{\includegraphics[width = 1\textwidth]{screenshot//recover_01.png}}
\caption{Usage: recover}
% \label{fig_process}
\end{figure}

Using \textit{recover 0}.

\begin{figure}[H]
\centerline{\includegraphics[width = 1\textwidth]{screenshot//recover_02.png}}
\caption{Usage: recover}
% \label{fig_process}
\end{figure}

It appears again in directory \textit{./dfs/}

\begin{figure}[H]
\centerline{\includegraphics[width = 1\textwidth]{screenshot//recover_03.png}}
\caption{Usage: recover}
% \label{fig_process}
\end{figure}

Finally, we exit Mini-DFS.

\begin{figure}[H]
\centerline{\includegraphics[width = 1\textwidth]{screenshot//quit_01.png}}
\caption{Usage: quit}
% \label{fig_process}
\end{figure}

% \begin{figure}[h]
% \centerline{\includegraphics[width = 1\textwidth]{screenshot//2_2.png}}
% \caption{Total early-stage Entrepreneurial Activity (TEA) Rates among Adults (ages 18-64) in 487 Economies, in Four Geographic Regions}
% \label{fig_TEA_global}
% \end{figure}

% \section{Pod}
% \subsection{1 Pod with 1 Container}
%
% We can see after creating pod1 by pod1.yaml, we can execute any command by kubectl exec -it pod1 -- command.
%
% \begin{figure}[H]
% \centerline{\includegraphics[width = 0.7\textwidth]{screenshot//1.png}}
% \caption{1 Pod with 1 Container}
% % \label{fig_1pod1container}
% \end{figure}
%
% \subsection{1 Pod with 2 Containers}
%
% We can see after we change index.html in container ct-debian, we can also see the change in container ct-nginx.
%
% \begin{figure}[H]
% \centerline{\includegraphics[width = 0.7\textwidth]{screenshot//2_1.png}}
% \caption{1 Pod with 2 Container}
% % \label{fig_1pod1container}
% \end{figure}






\end{document}
